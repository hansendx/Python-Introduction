\documentclass{article}
\usepackage[utf8]{inputenc}
\usepackage{tabularx}
\usepackage{booktabs}
\usepackage{hyperref}
\usepackage{listings}


%Style
\usepackage{xcolor}
 
\definecolor{codegreen}{rgb}{0,0.6,0}
\definecolor{codegray}{rgb}{0.5,0.5,0.5}
\definecolor{codepurple}{rgb}{0.58,0,0.82}
\definecolor{backcolour}{rgb}{0.95,0.95,0.92}
 
\lstdefinestyle{mystyle}{
    backgroundcolor=\color{backcolour},   
    commentstyle=\color{codegreen},
    keywordstyle=\color{magenta},
    numberstyle=\tiny\color{codegray},
    stringstyle=\color{codepurple},
    basicstyle=\ttfamily\footnotesize,
    breakatwhitespace=false,         
    breaklines=true,                 
    captionpos=b,                    
    keepspaces=true,                 
    numbers=left,                    
    numbersep=5pt,                  
    showspaces=false,                
    showstringspaces=false,
    showtabs=false,                  
    tabsize=2
}
 
\lstset{style=mystyle}
%Style


\setlength{\parindent}{0pt}

\title{Python Tutorial}
\author{Dominique Hansen}
\date{\today}

\begin{document}

\maketitle

\section{Python compared to other languages}

\subsection{Python compared to R}

\begin{table}[h]
\begin{tabularx}{\linewidth}{>{\parskip1ex}X@{\kern4\tabcolsep}>{\parskip1ex}X}
\toprule
\hfil\bfseries Python
&
\hfil\bfseries R
\\\cmidrule(r{3\tabcolsep}){1-1}\cmidrule(l{-\tabcolsep}){2-2}

%% PROS, seperated by empty line or \par
Was deveoped as teaching tool for programming.

Is a general purpose scripting language.

Is itself programmed in C and Python.

Main package repository is \href{https://pypi.org/}{pypi} with 216.195 packages.

Has a \href{https://github.com/mre/awesome-static-analysis\#python}{large library}
of software that helps developers write good code.

Has \href{https://www.python.org/dev/peps/pep-0008/}{official best practices}
to help in writing more comprehensible code.

Has a unittesting library in its own standard library.

&

%% CONS, seperated by empty line or \par
Was developed by statisticians for statisticians.

Is a domain specific scripting language.

Is itself programmed in C, Fortran and R.

Main package repository is \href{https://cran.r-project.org/}{CRAN} with 15.330 packages.

Has a \href{https://github.com/mre/awesome-static-analysis\#r}{small library}
of software that helps developers write good code.

Has no consensus on style or best practices.

Has third party libraries for unittesting.

\\\bottomrule
\end{tabularx}
\end{table}

\section{Pythonsyntax}



\section{Statements and Expressions}
\section{Decisions and Loops}
\section{Data Types, Sequences, Sets and Dictionaries}

The type of a variable is determined by the python interpreter at the time
of execution.
The interpreter can convert values to different types when it is necessary. 

\subsection{Build-in Types}

Python has extensive \href{https://docs.python.org/3/library/stdtypes.html}{documentation}
on their build-in types. 
The most basic types you will most commonly come in contact with:

\begin{itemize}
    \item integer
    \item float
    \item boolean
    \item string
\end{itemize}

These are immutable, i.e. their value cannot be changed.
Operations performed on them always create completely new "thing".


%http://pythontutor.com/visualize.html#code=NUMBER%20%3D%202%0AADD_RESULT%20%3D%20NUMBER%20%2B%202%0ADIV_RESULT%20%3D%20ADD_RESULT%20/%20NUMBER%0AMULT_RESULT%20%3D%20DIV_RESULT%20*%203%0AMODULO_RESULT%20%3D%20%28MULT_RESULT%20%2B%201%29%20%25%202%0A_STRING%20%3D%20%22HELLO%20WORLD%22%0A_LIST%20%3D%20list%28%29%0A_LIST2%20%3D%20%5BNUMBER,%20_STRING%5D%0A_LIST.append%28NUMBER%29%0A_LIST.append%28ADD_RESULT%29%0A_LIST.append%28_LIST2%29%0A_LIST2%5B0%5D%20%3D%20ADD_RESULT%0A_TUPLE%20%3D%20%281,2,3%29%0A_LIST%5B2%5D%20%3D%20_TUPLE%0A_TUPLE%20%3D%20_TUPLE%20%2B%20_TUPLE%0A_LIST%5B2%5D%20%3D%20_LIST2%0A_LIST2%20%3D%20_LIST2.append%28_LIST2%29&cumulative=false&curInstr=17&heapPrimitives=nevernest&mode=display&origin=opt-frontend.js&py=3&rawInputLstJSON=%5B%5D&textReferences=false

\subsection{Lists}

As the name suggests, this is a list of "things".
It can contain everything that can also be assigned to a variable.
Lists are mutable. They can be changed.
Elements can be added or removed from a list after it has been created.

Lists can be created empty:

\begin{lstlisting}[language=Python]
an_empty_list = list()
another_empty_list = []
\end{lstlisting}

and then be filled:

\begin{lstlisting}[language=Python]
list_for_content = an_empty_list
list_for_content.append(1)
# The name of an_empty_list is now a lie.
list_for_content.append(2)
list_for_content.append(3)
\end{lstlisting}

It can also be created with content:

\begin{lstlisting}[language=Python]
a_list_with_content = [1 ,2 ,3]
\end{lstlisting}

Elements of a list can be accessed like this:

\begin{lstlisting}[language=Python]
an_empty_list[0]
an_empty_list[2]
\end{lstlisting}

The index of lists in Python starts at 0 meaning that
$an\_empty\_list[0]$ will be $1$ and $an\_empty\_list[2]$
will be 3.

\href{http://pythontutor.com/visualize.html\#code=an\_empty\_list\%20\%3D\%20list\%28\%29\%0Aanother\_empty\_list\%20\%3D\%20\%5B\%5D\%0A\%0Alist\_for\_content\%20\%3D\%20an\_empty\_list\%0Alist\_for\_content.append\%281\%29\%0A\%23\%20The\%20name\%20of\%20an\_empty\_list\%20is\%20now\%20a\%20lie.\%0Alist\_for\_content.append\%282\%29\%0Alist\_for\_content.append\%283\%29\%0A\%0Aa\_list\_with\_content\%20\%3D\%20\%5B1\%20,2\%20,3\%5D\%0A\%0Aprint\%28an\_empty\_list\%29\%0Aprint\%28another\_empty\_list\%29\%0Aprint\%28list\_for\_content\%29\%0Aprint\%28a\_list\_with\_content\%29\%0A\%0Aprint\%28an\_empty\_list\%5B0\%5D\%29\%0Aprint\%28an\_empty\_list\%5B2\%5D\%29\&cumulative=false\&curInstr=13\&heapPrimitives=nevernest\&mode=display\&origin=opt-frontend.js\&py=3\&rawInputLstJSON=\%5B\%5D\&textReferences=false}{Here is an executable example.}


%([&%$#_{}~^\\]) \\\\\\1

%http://pythontutor.com/visualize.html#code=an_empty_list%20%3D%20list%28%29%0Aanother_empty_list%20%3D%20%5B%5D%0A%0Alist_for_content%20%3D%20an_empty_list%0Alist_for_content.append%281%29%0A%23%20an_empty_list%20name%20is%20now%20a%20lie%0Alist_for_content.append%282%29%0Alist_for_content.append%283%29%0A%0Aprint%28list_for_content%29%0Aprint%28an_empty_list%29&cumulative=false&curInstr=8&heapPrimitives=nevernest&mode=display&origin=opt-frontend.js&py=3&rawInputLstJSON=%5B%5D&textReferences=false

\subsubsection{Slicing}



\subsubsection{List comprehension}

\section{Functions}
\section{Generators and Iterators}
\section{Classes}
\section{Objects}
\section{Dealing with errors Error}
\section{Input and Output}
\section{Modules and Packages}
\section{Scope}
\section{String Processing}
\section{Processing Data Formats}


\end{document}
