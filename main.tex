\documentclass{article}
\usepackage[utf8]{inputenc}
\usepackage{tabularx}
\usepackage{booktabs}
\usepackage{hyperref}
\usepackage{listings}


%Style
\usepackage{xcolor}
 
\definecolor{codegreen}{rgb}{0,0.6,0}
\definecolor{codegray}{rgb}{0.5,0.5,0.5}
\definecolor{codepurple}{rgb}{0.58,0,0.82}
\definecolor{backcolour}{rgb}{0.95,0.95,0.92}
 
\lstdefinestyle{mystyle}{
    backgroundcolor=\color{backcolour},   
    commentstyle=\color{codegreen},
    keywordstyle=\color{magenta},
    numberstyle=\tiny\color{codegray},
    stringstyle=\color{codepurple},
    basicstyle=\ttfamily\footnotesize,
    breakatwhitespace=false,         
    breaklines=true,                 
    captionpos=b,                    
    keepspaces=true,                 
    numbers=left,                    
    numbersep=5pt,                  
    showspaces=false,                
    showstringspaces=false,
    showtabs=false,                  
    tabsize=2
}
 
\lstset{style=mystyle}
%Style


\setlength{\parindent}{0pt}
\setcounter{section}{-1}

\title{Python Tutorial}
\author{Dominique Hansen}
\date{\today}

\begin{document}

\maketitle

\section{Getting Started}

\subsection{Installing Python}

\subsubsection{Linux}
\label{linux_python_install}
Many linux systems have Python already installed.
Older systems might have Python2 alongside Python3.
Refrain from using Python2 since it is not fully compatible with Python3
and reached its end of life with the start of 2020.

On debian based systems Python3 can be installed with apt in the terminal:

\begin{lstlisting}[language=Bash]
sudo apt-get install python3.8
# Or just 3 if 3.8 is not available for you
sudo apt-get install python3
\end{lstlisting}

\subsubsection{Windows 10}
Windows 10 supports running linux inside itself with the
"Windows Subsystem for Linux". \footnote{\href{https://docs.microsoft.com/en-us/windows/wsl/install-win10}{Installation Guide}}
If you choose a debian based subsystem like ubuntu, you can use the
installation command from the section \ref{linux_python_install}.

\subsection{Windows 7}
If still have a Windows 7 installion on yout personal PC you should upgrade to Windows 10.
Windows 7 is no longer save to work in.
If you have to work with a Windows 7 systems on your work station at the DIW you can use conda.
You can get an installer for anaconda \href{https://www.anaconda.com/distribution/#download-section}{here}.
Anaconda can be installed without Admin privileges if you only install it for your user.
Install the full anaconda distribution not miniconda.




\subsection{Your Development Environment}

Probably the most popular editors for python developers are
\href{https://code.visualstudio.com/}{Visual Studio Code}
and \href{https://www.jetbrains.com/pycharm/}{PyCharm}.
PyCharm is specifically geared towards development in Python.
VSCode was created for web development but it also has a wide array of
extensions that allow you to work with other languages.
The Python extension is very mature and actively developed by microsoft.
VSCode is completely open source whereas PyCharm is partially open source, with
a free and a commercial version.

\subsubsection{VSCode Setup}

Get an installer for your system at
\href{https://code.visualstudio.com/}{code.visualstudio.com}.
Once installed open VSCode and install the Python extension.
\begin{itemize}
    \item Extensions can be opened via the toolbar on the left or with Ctrl+Shift+x.
    \item Search for Python in the extensions, first result should be the correct one.
    \item Open settings. Shortcut here is Ctrl+,
    \item On Windows search the settings for the entry
    "Python: Conda Path" and set it for your system.\footnote{Will Probably be at\\
D:{\textbackslash
}Users{\textbackslash
}YOURUSERNAME{\textbackslash
}AppData{\textbackslash
}Local{\textbackslash
}Continuum{\textbackslash
}anaconda3{\textbackslash
}Scripts{\textbackslash
}conda.exe
        }
    \item Enable all linters and style checker, that you like.
\end{itemize}

\subsubsection{Linters, Stylecheckers and Formatters}

A Linter is a program, that "looks" at your code and warns you about potential
problems with it. These Problems include:
\begin{itemize}
    \item Not adhering to a styleguide.
    \item Not adhering to conventions.
    \item Your code contains structures that are commonly associated with errors.
    \item Your code contains structures that make it harder to maintain.
    \item Your code contains actual errors, like syntax errors, that would crash it on execution.
\end{itemize}

Popular linters for Python are \href{https://github.com/PyCQA/pylint}{pylint} and
\href{https://github.com/PyCQA/pyflakes}{pyflakes}.
Pylint is more feature rich than pyflakes.
Some might find its features overwhelming and its problem finding too nit-picky.



\section{Python compared to other languages}

\subsection{Python compared to R}

\begin{table}[h]
\begin{tabularx}{\linewidth}{>{\parskip1ex}X@{\kern4\tabcolsep}>{\parskip1ex}X}
\toprule
\hfil\bfseries Python
&
\hfil\bfseries R
\\\cmidrule(r{3\tabcolsep}){1-1}\cmidrule(l{-\tabcolsep}){2-2}

%% PROS, seperated by empty line or \par
Was deveoped as teaching tool for programming.

Is a general purpose scripting language.

Is itself programmed in C and Python.

Main package repository is \href{https://pypi.org/}{pypi} with 216.195 packages.

Has a \href{https://github.com/mre/awesome-static-analysis\#python}{large library}
of software that helps developers write good code.

Has \href{https://www.python.org/dev/peps/pep-0008/}{official best practices}
to help in writing more comprehensible code.

Has a unittesting library in its own standard library.

&

%% CONS, seperated by empty line or \par
Was developed by statisticians for statisticians.

Is a domain specific scripting language.

Is itself programmed in C, Fortran and R.

Main package repository is \href{https://cran.r-project.org/}{CRAN} with 15.330 packages.

Has a \href{https://github.com/mre/awesome-static-analysis\#r}{small library}
of software that helps developers write good code.

Has no consensus on style or best practices.

Has third party libraries for unittesting.

\\\bottomrule
\end{tabularx}
\end{table}

\section{Pythonsyntax}

Most languages use visible characters to define a
\href{https://en.wikipedia.org/wiki/Block_(programming)}{code block}.

A loop in R can look like this:

\begin{lstlisting}[language=R]
for(number in 1:10) {
print(number)
}
\end{lstlisting}

And the same loop can also look like this:

\begin{lstlisting}[language=R]
for(number in 1:10){print(number)}
\end{lstlisting}

Or like this:

\begin{lstlisting}[language=R]
for(number in 1:10) {
    print(number)
}
\end{lstlisting}


The whitespace in the third example carries no syntactical meaning.
But this kind of indention is used by many programmers to make their
code more readable.
It makes it easier for people to determine the block a line of code belongs to.
Python on the other hand uses whitespace explicitly to denote a block.
The same loop in python would look like this:

\begin{lstlisting}[language=Python]
for number in range(1, 11):
    print(number)
\end{lstlisting}

Python forces its users to write cleaner looking code through its use of whitespace
as syntactical characters.
As a teaching language it was created with readability of the code in mind.

Python does not care if you use spaces or tab characters as long as it is consistent.
It also does not care how many spaces or tabs are used as long as it is consistent.
As with many scripting languages Python leaves some freedom to its users.
It is however strongly recommended\footnote{
    Python has a Style Guide
\href{https://www.python.org/dev/peps/pep-0008/}{PEP 8}
}
to limit oneself to the use of spaces
and exactly 4 spaces per indentation level.
Adhering to the style guide makes it easier for others (yourself in the future included)
to reuse the code.




\section{Statements and Expressions}
\section{Decisions and Loops}

\subsection{if, elif, else}

Often times we have to handle data differently during computation depending on
the makeup of the data.
Or we have to decide to do certain things if our data is a certain way.

To do this we can make use of the $if$, $if\cdots else$ and $if\cdots elif$
statements:

\begin{lstlisting}[language=Python]
if a_variable == 1:
    print("It is one.")
elif a_variable == 2:
    print("It is one.")
else:
    print("It is too much man.")
\end{lstlisting}

If the statement at $if$ evaluates to True "This is one." will be printed.
If it evaluates to False, the $elif$ statement is checked.
An $if$ statement can be followed by an arbitrary amount of $elif$ statements.
Only the one that evaluates to True gets its block executed, all statements
after that will be skipped.
The $else$ block will only be executed if all other checks evaluate to False.

\subsubsection{What is True?}

Obviously boolean $True$ and $False$ values evaluate to what they say.
But in python there are also "Truthy" and "Falsy" values.
These values are not of the type boolean but will be treated as such in the right context:

\begin{lstlisting}[language=Python]
a_list = []

if a_list:
    print("The first element is: " + a_list[0])
else:
    print("The list is empty.")
\end{lstlisting}

\begin{itemize}
    \item An empty list, dictionary, set, string or a 0 will evaluate to False in a boolean context.
    \item The special $None$ value will also evaluate to False. 
    \item A list, dictionary or set with at least one element will evaluate to True.
    \item All numbers other than 0 and a string with at least one character will also evaluate to True.
\end{itemize}






\section{Data Types, Sequences, Sets and Dictionaries}

The type of a variable is determined by the python interpreter at the time
of execution.
The interpreter can convert values to different types when it is necessary. 

\subsection{Build-in Types}

Python has extensive \href{https://docs.python.org/3/library/stdtypes.html}{documentation}
on their build-in types. 
The most basic types you will most commonly come in contact with:

\begin{itemize}
    \item integer
    \item float
    \item boolean
    \item string
\end{itemize}

These are immutable, i.e. their value cannot be changed.
Operations performed on them always create completely new "thing".


%http://pythontutor.com/visualize.html#code=NUMBER%20%3D%202%0AADD_RESULT%20%3D%20NUMBER%20%2B%202%0ADIV_RESULT%20%3D%20ADD_RESULT%20/%20NUMBER%0AMULT_RESULT%20%3D%20DIV_RESULT%20*%203%0AMODULO_RESULT%20%3D%20%28MULT_RESULT%20%2B%201%29%20%25%202%0A_STRING%20%3D%20%22HELLO%20WORLD%22%0A_LIST%20%3D%20list%28%29%0A_LIST2%20%3D%20%5BNUMBER,%20_STRING%5D%0A_LIST.append%28NUMBER%29%0A_LIST.append%28ADD_RESULT%29%0A_LIST.append%28_LIST2%29%0A_LIST2%5B0%5D%20%3D%20ADD_RESULT%0A_TUPLE%20%3D%20%281,2,3%29%0A_LIST%5B2%5D%20%3D%20_TUPLE%0A_TUPLE%20%3D%20_TUPLE%20%2B%20_TUPLE%0A_LIST%5B2%5D%20%3D%20_LIST2%0A_LIST2%20%3D%20_LIST2.append%28_LIST2%29&cumulative=false&curInstr=17&heapPrimitives=nevernest&mode=display&origin=opt-frontend.js&py=3&rawInputLstJSON=%5B%5D&textReferences=false

\subsection{Lists}

As the name suggests, this is a list of "things".
It can contain everything that can also be assigned to a variable.
Lists are mutable. They can be changed.
Elements can be added or removed from a list after it has been created.

Lists can be created empty:

\begin{lstlisting}[language=Python]
an_empty_list = list()
another_empty_list = []
\end{lstlisting}

and then be filled:

\begin{lstlisting}[language=Python]
list_for_content = an_empty_list
list_for_content.append(1)
# The name of an_empty_list is now a lie.
list_for_content.append(2)
list_for_content.append(3)
\end{lstlisting}

It can also be created with content:

\begin{lstlisting}[language=Python]
a_list_with_content = [1 ,2 ,3]
\end{lstlisting}

Elements of a list can be accessed like this:

\begin{lstlisting}[language=Python]
a_list_with_content[0]
a_list_with_content[2]
\end{lstlisting}

Indices in Python start at 0, meaning that
$an\_empty\_list[0]$ will be $1$ and $an\_empty\_list[2]$
will be 3.

\href{
    http://pythontutor.com/visualize.html\#code=an\_empty\_list\%20\%3D\%20list\%28\%29\%0Aanother\_empty\_list\%20\%3D\%20\%5B\%5D\%0A\%0Alist\_for\_content\%20\%3D\%20an\_empty\_list\%0Alist\_for\_content.append\%281\%29\%0A\%23\%20The\%20name\%20of\%20an\_empty\_list\%20is\%20now\%20a\%20lie.\%0Alist\_for\_content.append\%282\%29\%0Alist\_for\_content.append\%283\%29\%0A\%0Aa\_list\_with\_content\%20\%3D\%20\%5B1\%20,2\%20,3\%5D\%0A\%0Aprint\%28an\_empty\_list\%29\%0Aprint\%28another\_empty\_list\%29\%0Aprint\%28list\_for\_content\%29\%0Aprint\%28a\_list\_with\_content\%29\%0A\%0Aprint\%28an\_empty\_list\%5B0\%5D\%29\%0Aprint\%28an\_empty\_list\%5B2\%5D\%29\&cumulative=false\&curInstr=13\&heapPrimitives=nevernest\&mode=display\&origin=opt-frontend.js\&py=3\&rawInputLstJSON=\%5B\%5D\&textReferences=false
    }{Here is an executable example.}


%([&%$#_{}~^\\]) \\\\\\1

%http://pythontutor.com/visualize.html#code=an_empty_list%20%3D%20list%28%29%0Aanother_empty_list%20%3D%20%5B%5D%0A%0Alist_for_content%20%3D%20an_empty_list%0Alist_for_content.append%281%29%0A%23%20an_empty_list%20name%20is%20now%20a%20lie%0Alist_for_content.append%282%29%0Alist_for_content.append%283%29%0A%0Aprint%28list_for_content%29%0Aprint%28an_empty_list%29&cumulative=false&curInstr=8&heapPrimitives=nevernest&mode=display&origin=opt-frontend.js&py=3&rawInputLstJSON=%5B%5D&textReferences=false

\subsubsection{Slicing}

Slicing gives you an easy and comprehensible way to get a part of a list.
Slicing a list looks similar to addressing a single element:

\begin{lstlisting}[language=Python]
# The range() function is useful for the creation of numerical sequences.
n_up_to_a_hundred = list(range(1,101))
n_up_to_50 = n_up_to_a_hundred[:50]
multiples_of_two_up_to_50 = n_up_to_a_hundred[1:50:2]
\end{lstlisting}
The numbers between the brackets provide [first index:last index:step]


\subsubsection{List comprehension}

List comprehension, somewhat like slicing, lets us create a list from another list.
Slicing only allows us to select elements from a list by index.
List comprehension is far more powerful.



\section{Functions}

A function is a code structure, that accepts some form of input, does something
and returns a value.



\section{Generators and Iterators}
\label{generators}
\section{Classes and Objects}
\section{Dealing with Errors}
\section{Input and Output}
\section{Modules and Packages}
\section{Scope}
\section{String Processing}
\section{Processing Data Formats}


\end{document}
