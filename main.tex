\documentclass{article}
\usepackage[utf8]{inputenc}
\usepackage{tabularx}
\usepackage{booktabs}
\usepackage{hyperref}


\title{Python Tutorial}
\author{Dominique Hansen}
\date{\today}

\begin{document}

\maketitle

\section{Python compared to other languages}

\subsection{Python compared to R}

\begin{table}[h]
\begin{tabularx}{\linewidth}{>{\parskip1ex}X@{\kern4\tabcolsep}>{\parskip1ex}X}
\toprule
\hfil\bfseries Python
&
\hfil\bfseries R
\\\cmidrule(r{3\tabcolsep}){1-1}\cmidrule(l{-\tabcolsep}){2-2}

%% PROS, seperated by empty line or \par
Was deveoped as teaching tool for programming.

Is a general purpose scripting language.

Is itself programmed in C and Python.

Main package repository is \href{https://pypi.org/}{pypi} with 216.195 packages.

Has a \href{https://github.com/mre/awesome-static-analysis\#python}{large library}
of software that helps developers write good code.

Has \href{https://www.python.org/dev/peps/pep-0008/}{official best practices}
to help in writing more comprehensible code.

&

%% CONS, seperated by empty line or \par
Was developed by statisticians for statisticians.

Is a domain specific scripting language.

Is itself programmed in C, Fortran and R.

Main package repository is \href{https://cran.r-project.org/}{CRAN} with 15.330 packages.

Has a \href{https://github.com/mre/awesome-static-analysis\#r}{small library}
of software that helps developers write good code.

Has no consensus on style or best practices.

\\\bottomrule
\end{tabularx}
\end{table}

\section{Pythonsyntax}



\section{Statements and Expressions}
\section{Decisions and Loops}
\section{Data Types, Sequences, Sets and Dictionaries}

\subsection{Build-in Types}

Python has extensive \href{https://docs.python.org/3/library/stdtypes.html}{documentation}
on their build-in types. 
The most basic types you will most commonly come in contact with:


\begin{itemize}
    \item integer
    \item float
    \item boolean
    \item string
\end{itemize}

These are immutable, i.e. their value cannot be changed.

%http://pythontutor.com/visualize.html#code=NUMBER%20%3D%202%0AADD_RESULT%20%3D%20NUMBER%20%2B%202%0ADIV_RESULT%20%3D%20ADD_RESULT%20/%20NUMBER%0AMULT_RESULT%20%3D%20DIV_RESULT%20*%203%0AMODULO_RESULT%20%3D%20%28MULT_RESULT%20%2B%201%29%20%25%202%0A_STRING%20%3D%20%22HELLO%20WORLD%22%0A_LIST%20%3D%20list%28%29%0A_LIST2%20%3D%20%5BNUMBER,%20_STRING%5D%0A_LIST.append%28NUMBER%29%0A_LIST.append%28ADD_RESULT%29%0A_LIST.append%28_LIST2%29%0A_LIST2%5B0%5D%20%3D%20ADD_RESULT%0A_TUPLE%20%3D%20%281,2,3%29%0A_LIST%5B2%5D%20%3D%20_TUPLE%0A_TUPLE%20%3D%20_TUPLE%20%2B%20_TUPLE%0A_LIST%5B2%5D%20%3D%20_LIST2%0A_LIST2%20%3D%20_LIST2.append%28_LIST2%29&cumulative=false&curInstr=17&heapPrimitives=nevernest&mode=display&origin=opt-frontend.js&py=3&rawInputLstJSON=%5B%5D&textReferences=false

\subsection{Lists}

As the name suggests, this is a list of "things".
It can contain everything that can also be assigned to a variable.

\subsubsection{Slicing}

\subsubsection{List comprehension}

\section{Functions}
\section{Generators and Iterators}
\section{Classes}
\section{Objects}
\section{Dealing with errors Error}
\section{Input and Output}
\section{Modules and Packages}
\section{Scope}
\section{String Processing}
\section{Processing Data Formats}


\end{document}
